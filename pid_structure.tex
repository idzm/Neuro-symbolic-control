\documentclass[tikz, border=10pt]{standalone}
\usepackage[T2A]{fontenc}
\usepackage[utf8]{inputenc}
\usepackage[russian]{babel}
\usepackage{amsmath}
\usetikzlibrary{shapes, arrows, positioning, calc}

\begin{document}

\begin{tikzpicture}[
    auto, 
    node distance=2cm, 
    >=latex',
    label_node/.style={inner sep=3pt},
    block/.style={
        draw, 
        fill=white, 
        rectangle, 
        minimum height=1.1cm, 
        minimum width=3.5cm, 
        text width=3.2cm,    
        align=left,           
        inner xsep=5pt        
    },
    sum/.style={draw, fill=white, circle, inner sep=2pt}
]

    % Узлы
    \node (input_text) {задание};
    \node [sum, right=1.5cm of input_text] (sum1) {\Large $\Sigma$};
    
    % Точка разветвления ошибки
    \node [coordinate, right=1cm of sum1] (branch) {};
    
    % Блоки P, I, D
    \node [block, right=2.5cm of sum1] (integral) {
        \textbf{I} \hfill $K_i \int_{0}^{t} e(t) dt$
    };
    \node [block, above=0.6cm of integral] (prop) {
        \textbf{P} \hfill $K_p e(t)$
    };
    \node [block, below=0.6cm of integral] (deriv) {
        \textbf{D} \hfill $K_d \frac{de}{dt}$
    };
    
    % Сумматор после регулятора и Процесс
    \node [sum, right=1.5cm of integral] (sum2) {\Large $\Sigma$};
    \node [block, right=1.0cm of sum2, minimum width=2.0cm, text width=1.8cm, align=center] (process) {\textbf{процесс}};
    
    % Текст выхода и точка разветвления обратной связи
    \node [label_node, right=0.8cm of process] (out_text) {выход};
    \node [coordinate] (feedback_node) at ($(out_text.east) + (0.3,0)$) {};

    % Связи

    % Входной сигнал
    \draw [->] (input_text) -- node[above, pos=0.9] {$+$} (sum1);
    
    % Сигнал ошибки и разветвление на P, I, D
    \node [fill=white, inner sep=2pt] (err_text) at (branch) {ошибка};
    \draw [->] (sum1) -- (err_text);
    
    \draw [->] (err_text) -- (integral);
    \draw [->] (err_text) |- (prop);
    \draw [->] (err_text) |- (deriv);
    
    % Объединение сигналов в sum2
    \draw [->] (prop) -| (sum2);
    \draw [->] (integral) -- (sum2);
    \draw [->] (deriv) -| (sum2);
    
    % От регулятора к процессу и на выход
    \draw [->] (sum2) -- (process);
    \draw [->] (process) -- (out_text);
    
    % Завершающая стрелка выхода
    \draw (out_text) -- (feedback_node);
    \draw [->] (feedback_node) -- ++(1,0);
    
    % Линия обратной связи
    \draw [->] (feedback_node) |- ++(0,-3.2) -| node[left, pos=0.97] {$-$} (sum1);

\end{tikzpicture}

\end{document}