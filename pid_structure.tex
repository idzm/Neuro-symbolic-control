\documentclass[tikz, border=10pt]{standalone}
\usepackage[T2A]{fontenc}
\usepackage[utf8]{inputenc}
\usepackage[russian]{babel}
\usepackage{amsmath}
\usetikzlibrary{shapes, arrows, positioning, calc}

\begin{document}

\begin{tikzpicture}[
    auto, 
    node distance=1.5cm, 
    >=latex',
    signal_label/.style={fill=white, inner sep=1.5pt, font=\small},
    block/.style={
        draw, 
        fill=white, 
        rectangle, 
        minimum height=0.9cm, 
        minimum width=2.4cm, 
        text width=2.2cm,    
        align=center,          
        inner xsep=3pt,
        font=\small
    },
    process_block/.style={
        block,
        minimum width=1.6cm, 
        text width=1.4cm
    },
    sum/.style={draw, fill=white, circle, inner sep=1pt}
]

    % --- Nodes ---
    
    \node [signal_label] (input_text) {задание};
    \node [sum, right=0.6cm of input_text] (sum1) {\Large $\Sigma$};
    
    % Уменьшено общее расстояние между сумматорами для компактности
    \node [sum, right=5.5cm of sum1] (sum2) {\Large $\Sigma$}; 

    % Точка разветвления (ошибка)
    \node [coordinate, right=1cm of sum1] (branch) {};

    % Размещение блоков ровно посередине между branch и sum2
    \node [block] (integral) at ($(branch)!0.5!(sum2)$) {
        \textbf{I} \hfill $K_i \int_{0}^{t} e(t) dt$
    };
    
    % Уменьшен вертикальный отступ между блоками (1.2 вместо 1.5)
    \node [block] (prop) at ($(integral.center) + (0, 1.2)$) {
        \textbf{P} \hfill $K_p e(t)$
    };
    
    \node [block] (deriv) at ($(integral.center) + (0, -1.2)$) {
        \textbf{D} \hfill $K_d \frac{de}{dt}$
    };
    
    % Процесс и выход расположены ближе
    \node [process_block, right=0.6cm of sum2] (process) {\textbf{процесс}};
    \node [signal_label, right=0.6cm of process] (out_text) {выход};
    
    \node [coordinate] (end_point) at ($(out_text.east) + (0.6,0)$) {};
    \node [coordinate] (feedback_node) at ($(out_text.east)!0.5!(end_point)$) {};

    % --- Connections ---

    \draw [->] (input_text) -- node[above, pos=0.8, font=\footnotesize] {$+$} (sum1);
    
    \node [signal_label] (err_text) at (branch) {ошибка};
    \draw [->] (sum1) -- (err_text);
    
    \draw [->] (err_text) -- (integral);
    \draw [->] (err_text) |- (prop);
    \draw [->] (err_text) |- (deriv);
    
    \draw [->] (prop) -| (sum2);
    \draw [->] (integral) -- (sum2);
    \draw [->] (deriv) -| (sum2);
    
    \draw [->] (sum2) -- (process);
    \draw [->] (process) -- (out_text);
    
    \draw [->] (out_text) -- (end_point);
    % Линия обратной связи стала компактнее по высоте
    \draw [->] (feedback_node) |- ++(0,-2.0) -| node[left, pos=0.98, font=\footnotesize] {$-$} (sum1);

\end{tikzpicture}

\end{document}