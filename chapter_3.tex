\chapter{Онтологические модели управления технологическими процессами}

\section{Общее описание}

Важную роль играют все участники процесса: пользователи — люди (операторы, мастера, начальники цеха и т.п.); устройства — датчики и исполнительные механизмы (температурные датчики, насосы, клапана и т.п.); механизированные системы — конвейерные системы, агрегаты; роботизированные системы — шарнирные роботы, дельта-роботы, манипуляторы; так и программные системы — SCADA, MES, ERP. Их взаимодействие обеспечивает достижение поставленной цели, устранение и предотвращение внештатных ситуаций. Причем важное влияние имеют как количественные показатели (количество операторов, устройств, агрегатов, панелей управления и т.п.), так и качество (качество устройств, квалификация операторов, качество программных систем и т.п.). В системах управления также важным является такой параметр, как скорость принятия решений — оперативное внесение изменений для выполнения поставленных планов. Онтологическое описание позволяет получить общее описание, понятное всем данным участникам процесса.
Ключевым фактором, препятствующим появлению систем автономного интеллектуального производства уже сегодня, является отсутствие общего архитектурного подхода к созданию цифровых платформ управления производствами, с одной стороны, и «зоопарк» форматов и стандартов работы с данными — с другой. Примечательно, что на эти два вызова уже существуют технологические ответы, которые получили общее название «индустриальные графы знаний» \cite{Муромцев2019} и показывают преимущества онтологического подхода, обеспечивающего гибкое моделирование и интероперабельность данных, стек семантических технологий, позволяющий выполнять анализ неструктурированной информации и интеллектуальный поиск данных во множестве разнородных источников, а также машинное обучение, обеспечивающее анализ и классификацию данных, в том числе в условиях неполной информации.

Анализ работ позволил сформулировать наиболее важные и распространенные проблемы, связанные с разработкой и применением современных стандартов в различных областях \cite{Серенков2004, Углев2012}:

\begin{itemize}
    \item Прежде всего, сложность ведения самих стандартов из-за дублирования информации, особенно сложности изменения терминологии.
    \item Дублирование информации в документации, описывающей стандарт.
    \item Проблемы интернационализации стандартов — перевод стандарта на несколько языков фактически требует поддержки и согласования независимых версий стандарта на разных языках.
    \item Как следствие — несоответствия в формате разных стандартов. Также усложняется автоматизация процесса разработки и применения стандартов.
    \item Неудобство использования стандарта, особенно сложность поиска нужной информации. Как следствие — сложность изучения стандартов.
    \item Сложность автоматизации проверки соответствия объекта или процесса требованиям конкретного стандарта.
    \item и т. д.
\end{itemize}

\section{Краткая характеристика ISA стандартов в АСУТП}

Эти проблемы в основном связаны с представлением стандартов. Наиболее перспективным подходом к решению этих проблем является преобразование каждого конкретного стандарта в базу знаний, которая базируется на наборе онтологий, соответствующих этому стандарту \cite{Серенков2004, Углев2012}. Такой подход позволяет существенно автоматизировать процессы разработки стандарта и его применения.

В качестве примера рассмотрим стандарт \textbf{\textit{ISA-88}} \cite{ISA88} (базовый стандарт для серийного производства). Хотя этот стандарт широко используется американскими и европейскими компаниями и активно внедряется на территории Республики Беларусь, он имеет ряд недостатков. Основными стандартами серийных систем ISA являются:

\begin{itemize}
    \item \textit{ANSI/ISA-88.00.01-2010}, Batch Control -- Part 1: Models and Terminology;
    \item \textit{ISA-88.00.02-2001}, Batch Control -- Part 2: Data Structures and Guidelines for Languages;
    \item \textit{ANSI/ISA-TR88.00.02-2015}, Machine and Unit States: An implementation example of \textit{ANSI/ISA-88.00.01};
    \item \textit{ISA-88.00.03-2003}, Batch Control -- Part 3: General and Site Recipe Models and Representation;
    \item \textit{ISA-TR88.0.03-1996}, Possible Recipe Procedure Presentation Formats;
    \item \textit{ANSI/ISA-88.00.04-2006}, Batch Control -- Part 4: Batch Production Records;
    \item \textit{ISA-TR88.95.01-2008}, Using ISA-88 and ISA-95 Together;
    \item \textit{IEC 61512-1}, The European version approved in 1997, based on the older version \textit{ISA-88.01-1995};
    \item \textit{GOST R IEC 61512-1-2016} -- Russian version of the standard, identical to \textit{IEC 61512-1}.
\end{itemize}

Другим стандартом, часто используемым в контексте Industry 4.0, является \textit{\textbf{ISA-95}} \cite{ISA95}. \textit{\textbf{ISA-95}} — это отраслевой стандарт для описания систем управления высокого уровня. Его основная цель — упростить разработку таких систем, абстрагироваться от аппаратной реализации и предоставить единый интерфейс для взаимодействия со слоями ERP и MES. Состоит из следующих частей:

\begin{itemize}
    \item \textit {ANSI/ISA-95.00.01-2010}, Enterprise-Control System Integration -- Part 1: Models and Terminology;
    \item \textit {ANSI/ISA-95.00.02-2018}, Enterprise-Control System Integration -- Part 2: Objects and Attributes for Enterprise-Control System Integration;
    \item \textit {ANSI/ISA-95.00.03-2013}, Enterprise-Control System Integration -- Part 3: Activity Models of Manufacturing Operations Management;
    \item \textit {ANSI/ISA-95.00.04-2018}, Enterprise-Control System Integration -- Part 4: Objects and Attributes for Manufacturing Operations Management Integration;
    \item \textit {ANSI/ISA-95.00.05-2018}, Enterprise-Control System Integration -- Part 5: Business-to-Manufacturing Transactions;
    \item \textit {ANSI/ISA-95.00.06-2014}, Enterprise-Control System Integration -- Part 6: Messaging Service Model;
    \item \textit {ANSI/ISA-95.00.07-2017}, Enterprise-Control System Integration -- Part 7: Alias Service Model;
    \item \textit {ANSI/ISA-95.00.08-2020}, Enterprise-Control System Integration -– Part 8: Information Exchange Profiles.
\end{itemize}

Модели помогают определить границы между бизнес-системами и системами управления. Они помогают ответить на вопросы о том, какие функции могут выполнять какие задачи и какой информацией должны обмениваться приложения. Комитет по разработке стандартов ISA5 часто называют среди практиков стандартом ISA-5.1. Однако комитет ISA5 «Documentation of Measurement and Control Instruments and Systems» имеет более широкую сферу деятельности, а именно разработку стандартов, рекомендуемых практик и технических отчетов для документирования и иллюстрации измерительных и контрольных приборов и систем, подходящих для всех отраслей. Стандарты ISA5 состоят из следующих документов:

\begin{itemize}
    \item \textit {ANSI/ISA-5.1-2022}, Instrumentation Symbols and Identification;
    \item \textit {ISA-5.9 working group}, Controller Algorithms and Performance;
    \item \textit {ISA-5.4}, Instrument Loop Diagrams;
    \item \textit {ISA-5.5-1985}, Graphic Symbols for Process Displays;
    \item \textit {ISA-5.6}, Documentation for Control Software Applications;
\end{itemize}

Этот стандарт полезен, когда требуется ссылка на оборудование в химической, нефтяной, энергетической, кондиционирующей, металлообрабатывающей и многих других отраслях промышленности. Стандарт позволяет любому человеку с достаточным уровнем знаний о заводе читать технологические схемы, чтобы понимать, как измерять и контролировать процесс, не вдаваясь в подробности приборов или знания эксперта.

\section{Формализация стандарта ISA-88}

В системах управления знаниями с целью решения задач в проблемной области в соответствии с онтологическим подходом строится концептуальная модель этой проблемной области – предметная область [Голенков, 2013]. Для работы с ней строятся её спецификации – онтологии различного вида [Давыденко и др., 2016; Ивашенко, 2011], которые могут рассматриваться в качестве многократно используемых компонентов [Ивашенко, 2011; Голенков, 2013]. Онтологии, которые являются результатом согласования нескольких участников их разработки в рамках предприятия, могут быть отнесены к стандартам систем управления знаниями этого предприятия.
Системы управления знаниями могут использовать знания, соответствующие различным стандартам. Для систем управления знаниями серийного (рецептурного) производства существует несколько стандартов (IEC 61131/3, ISA-88, ISA-95, IEC-62264, IEC-60848, IEC-60050, IEC-19501, ISO/IEC-9075), которые относятся к разным уровням и аспектам подсистем подобных систем.
Любой стандарт можно рассматривать как онтологию, которая в свою очередь состоит из множества онтологий, которые могут быть структурированы как иерархия онтологий. В соответствии с моделью спецификации знаний [Ивашенко, 2015], может быть выстроена формальная модель онтологии [Гаврилова, 2000], соответствующая формальной модели стандарта. Выстроенная формальная модель онтологии задаёт онтологию, являющуюся результатом формализации такого стандарта, т.е. результатом «понимания» стандарта содержащей её системой. С целью формализации необходимо произвести анализ документа стандарта, сформировать первичную формальную модель онтологии, носитель которой включает значимые элементы исходного документа, а сигнатура – их отношения и функции, соответствующие пониманию этого стандарта в той или иной семантике. Далее, в соответствии с моделью спецификации знаний, необходимо построить соответствующие онтологию и её формальную модель в семантике системы управления знаниями предприятия. В случае предлагаемого подхода – в семантике модели унифицированного семантического представления знаний – модельной семантике ситуативных множеств [Ивашенко, 2013]. Так как выстраиваемая онтология рассматривается как часть базы знаний [Ивашенко, 2009, Давыденко и др., 2016] системы управления знаниями, то порядок формирования этих онтологий и их формальных моделей может соответствовать различным методикам [Ивашенко, 2011] и моделям проектирования [Давыденко, 2016] баз знаний.
В случае готового документа, такого как стандарт, можно использовать методику концептуального или структурного проектирования [Ивашенко, 2011].

В последнем случае формализация состоит из этапов следующих классов, которые соответствуют явным формулировкам задач процесса проектирования базы знаний:
\begin{itemize}
    \item анализ содержания документа, спецификация его структуры и выделение разделов и подразделов до атомарных (элементарных);
    \item формирование спецификации каждого раздела, описываемых в них (значимых, ключевых) элементов и соответствующих предметных областей. Каждая предметная область специфицируется набором онтологий, описывающих соответствующий вид знаний:
    \begin{itemize}
        \item терминологическая онтология;
        \item теоретико-множественная онтология;
        \item логическая онтология;
        \item онтология задач и решений задач;
        \item и другие;
    \end{itemize}
    \item оформление сформированных спецификаций предметных областей в виде разделов, верификация и интеграция их в общую иерархию структур (разделов) базы знаний.
\end{itemize}

Последние два класса этапов повторяются, пока все задачи не будут решены. Количество этапов и перечень соответствующих задач может уточняться в процессе формализации стандарта.
В результате выполнения первого этапа получена иерархия разделов, включающая декомпозицию на подразделы. Каждая часть стандарта соответствует своему разделу. Для каждого формализуемого раздела выделены его ключевые элементы, так например, раздел «Definitions» («Термины и определения») второй части документа «Data structures and Guidelines for Languages» («Структуры данных и руководство по языку») имеет не менее восьми ключевых элементов. Для каждого такого ключевого элемента указаны термины на соответствующих языках, например, «recipe entity» («рецептурная сущность») (Рис. \ref{fig:recipe_entity_main_scg}), в случае наличия синонимов, например, для «соединительное звено» («link»), указывается каждый из них (Рис. \ref {fig:recipe_entity_synonym_scg}). Способность предлагаемого подхода различать понятия и термины, позволяет легко адаптировать систему к разным языкам и конкретным пользователям или вносить изменения в стандарт, связанные с неоднозначностью употребления терминов на практике или сменой терминологии при обновлении стандарта.

\begin{figure}[H]
    \centering
    \includegraphics{images/chapter_3/рецептурная_сущность_главная_scg.jpg}
    \caption{Указание терминов ключевых элементов}
    \label{fig:recipe_entity_main_scg}
\end{figure}

\begin{figure}[H]
    \centering
    \includegraphics{images/chapter_3/рецептурная_сущность_синоним_scg.jpg}
    \caption{Указание синонимичных терминов}
    \label{fig:recipe_entity_synonym_scg}
\end{figure}

Другими ключевыми элементами в разделах документа являются:
\begin{itemize}
    \item обзорная модель (overview model),
    \item модель процессов (process model),
    \item модель технологического управления (procedural control model),
    \item модель рецептур (recipe model),
    \item физическая модель (physical model),
    \item модель оборудования (equipment model),
    \item модель деятельности управления (control activity model),
    \item планы рецептурного производства (batch schedule),
    \item история рецептурного производства (batch history) и отчёты (batch report),
    \item структуры обмена информацией (batch exchange table) и используемые типы данных,
    \item библиотеки технологических элементов (building block library and process element library):
          \begin{itemize}
              \item процессы (технологии),
              \item этапы процесса (процессы),
              \item операции процесса (производственные операции),
              \item действия процесса (фазы),
          \end{itemize}
    \item преобразования рецептур (transform components and transforming tasks),
    \item и другие.
\end{itemize}

Каждому выделенному ключевому элементу соответствует некоторая предметная область. Так, например, рецептурной сущности соответствует предметная область рецептов, а сущности оборудования соответствует предметная область оборудования.

На следующих этапах между представленными ключевыми элементами формируются структуры различного вида, в виде таксономий или реляционных структур, для этого используются обозначения соответствующих отношений, их связок и областей определения.

Предлагаемый подход позволяет интегрировать различные представления и модели, описывающие процессы и решения задач рецептурного производства, например, диаграммы переходов состояний в виде моделей ситуационного управления, указывающих режимы и описывающих состояния сущностей оборудования из модели оборудования и технологических элементов из модели процедурного управления, или процессная модель деятельности управления рецептурным производством.

На основе полученного в результате представления стандарта путём добавления агентов, способных решать задачи информационного поиска, можно строить интеллектуально справочные системы различного спектра, для этого необходимо добавить и использовать соответствующие уже разработанные агенты (например, поиска по образцу) в базу знаний. В этом случае системе управления знаниями предприятия, обладающей знаниями по стандарту ISA-88, можно задавать вопросы, в том числе связанные с поиском по образцу.

Рассмотрим некоторые аспекты построения онтологической модели предприятия рецептурного производства в соответствии со стандартом ISA-88.

\section{Модель оборудования предприятия}

В рамках перехода от традиционных АСУТП к интеллектуальным особый интерес представляет формализация документации предприятия, в частности технологических схем. Принципиальная простота взаимного перехода между технологическим чертежом и его семантическим представлением, дает возможность задавать вопросы, касающиеся элементов чертежа, инициировать команды, воздействующие на состояние соответствующего физического оборудования и отслеживать динамику состояния оборудования. Эта функциональность обеспечивается коллективом рецепторных и эффекторных агентов, работающих над семантическим представлением документации, хранящимся в общей семантической памяти. Это позволяет «оживить» техническую документацию предприятия, сделать ее многоцелевой. Таким образом, в зависимости от пользователя (оператор, мастер, начальник цеха) систем может давать нужные интеллектуальные ответы.

Модель оборудования строится на базе предметной области физических моделей рецептурных производств. Физическая модель (модель оборудования) в общем случае включает семь уровней:

\begin{itemize}
    \item блок управления (Control Module);
    \item агрегат (Equipment Module);
    \item установка (Unit);
    \item ячейка процесса (Process Cell);
    \item производственный участок (Area);
    \item производство (Site);
    \item предприятие (Enterprise).
\end{itemize}

Необходимо построить её структурную спецификацию на языке SCn.
